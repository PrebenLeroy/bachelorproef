%%=============================================================================
%% Voorwoord
%%=============================================================================

\chapter*{Woord vooraf}
\label{ch:voorwoord}

%% TODO:
%% Het voorwoord is het enige deel van de bachelorproef waar je vanuit je
%% eigen standpunt (``ik-vorm'') mag schrijven. Je kan hier bv. motiveren
%% waarom jij het onderwerp wil bespreken.
%% Vergeet ook niet te bedanken wie je geholpen/gesteund/... heeft
\begin{flushright}
	\textit{Mei, 2018}
\end{flushright}

Deze bachelorproef dient als sluitstuk van mijn driejarige opleiding Toegepaste Informatica aan Hogeschool Gent. Hiermee wil ik aantonen dat ik op een kritische manier een onderzoek kan uitvoeren naar een nog niet zo evident onderwerp in de IT. De IT-wereld is een wereld die altijd in verandering is. Daarom is het belangrijk dat men altijd op een kritische manier nieuwe technologieën onderzoekt.

Het onderzoek dat ik zal uitvoeren voor deze bachelorproef heeft te maken met artificiële intelligentie. AI speelt een belangrijke rol in ons dagelijks leven. In de toekomst zal AI zelfs nog belangrijker zijn dan vandaag. Denk hierbij maar aan zelfrijdende auto's of automatische stofzuigers en dergelijke. 

Door toedoen van artificiële intelligentie zal ook onze maatschappij grondig veranderen. Vele jobs zullen hierdoor verdwijnen. Ook in de wereld van de IT. Daarom vond ik het belangrijk hiernaar onderzoek te doen. Bestaat er een kans dat AI het werk van een IT'er gaat overnemen? In het bijzonder het werk van een programmeur? Dit ga ik trachten te onderzoeken. 

Graag zou ik mijn promotor, de heer Steven Van Impe, en mijn co-promotor, de heer Karel Serruys, willen bedanken om mij met raad en daad bij te staan doorheen dit proces. Terwijl de heer Van Impe mij vooral op technisch vlak ondersteunde, ondersteunde de heer Karel Serruys op inhoudelijk vlak. Hij was degene die mij in de juiste richting dreef voor het uitvoeren van dit onderzoek.

\break
Daarnaast zou ik graag mijn ouders bedanken om mij de afgelopen drie jaar te ondersteunen, zeker op momenten wanneer het moeilijk ging. Ook al weten of begrijpen ze niet met war ik bezig ben, toch probeerden zij mij bij te staan. Hierbij zou ik mij ook willen excuseren wanneer ik mij de afgelopen drie jaar irritant gedraagde wanneer het wat moeilijker ging. 

Ik wens iedere lezer veel leesplezier bij het lezen van deze bachelorproef.

\begin{flushright}
	\textit{Preben Leroy \\
		Bachelor Toegepaste Informatica, keuzepakket Mobile Applicaties \\
		Hogeschool Gent \\
		Academiejaar 2017-2018}
\end{flushright}