%%=============================================================================
%% Experiment
%%=============================================================================

\chapter{Experiment}
\label{ch:experiment}

In dit hoofstuk zal het complete experiment besproken worden. Eerst en vooral zal er gekeken worden naar de mogelijkheid om de algoritmen werkende te hebben. Indien alles goed zal verlopen, zullen de algoritmen uitgevoerd worden op andere datasets. Hierbij wordt dan gekeken of de bekomen resultaten gelijkaardig zijn aan de verkregen resultaten. 

Het gevolg van deze experimenten valt dan te lezen in het volgende hoofdstuk. Er wordt getracht een duidelijke conclusie te vormen die antwoord bied op de vooropgestelde onderzoeksvragen. Het kan natuurlijk ook anders uitdraaien. De algoritmen zouden bijvoorbeeld niet kunnen werken of de resultaten zouden niet kunnen kloppen. Dan zal er moeten gezocht worden naar een duidelijke reden waarom het niet werkt.

\section{WikiSQL}

\subsection{Algoritme}

Het WikiSQL algoritme, welke te verkrijgen is via de Github-repository van \textcite{wikisql}, biedt de gebruiker een geannoteerd semantisch sjabloon voor de ontwikkeling van natuurlijke taalinterfaces. WikiSQL is een dataset, ontworpen door Salesforce en gelijktijdig uitgebracht met hun werk over Seq2SQL. Seq2SQL is reeds besproken in hoofdstuk \ref{ch:stand-van-zaken}, sectie \ref{sec:Salesforce - Seq2SQL}. Dit algoritme wordt later in dit hoofdstuk ook nog kort aangehaald.

Naast Seq2SQL wordt de WikiSQL dataset ook veel gebruikt voor andere algoritmen met doel het natuurlijke taalprogrammering. In hoofdstuk \ref{ch:stand-van-zaken} zijn er reeds een aantal van besproken. Ook wordt bijvoorbeeld SQLNet (sectie \ref{sec:SQLNet}) getraind op basis van de WikiSQL dataset.

\subsection{Opbouw}

--Opbouw--

\subsection{Uitvoering}

--Uitvoering--

\subsection{Resultaten}

--Resultaten--

\section{SQLNet}

\subsection{Algoritme}

Het SQLNet algoritme (sectie \ref{sec:SQLNet}), welke te verkrijgen is via de Github-repository van \textcite{sqlnet}, biedt de gebruiker een neuraal netwerk aan voor de generatie van query's op basis van natuurlijke taal. Zoals reeds vermeld, wordt het systeem getraind op basis van de WikiSQL dataset. Daarnaast wordt er ook een implementatie gegeven voor het Seq2SQL algoritme voor het voorspellen van SQL-query's.

Het algoritme maakt geen gebruik van reïnforcement learning. Dit is het mechanisme dat ervoor zorgt dat het algoritme een soort beloning krijgt vanaf wanneer er een goede query wordt gegenereerd. Dit in tegenstelling tot Seq2SQL, waar reïnforcement learning weldegelijk wordt toegepast.

\subsection{Opbouw}

--Opbouw--

\subsection{Uitvoering}

--Uitvoering--

\subsection{Resultaten}

--Resultaten--

\section{Seq2SQL}

\subsection{Algoritme}

Het Seq2SQL algoritme (sectie \ref{sec:Salesforce - Seq2SQL}), welke te verkrijgen is via de Github-repository van \textcite{seq2sql}, biedt de gebruiker een neuraal netwerk aan voor de generatie van query's op basis van natuurlijke taal. Dit algoritme is ontwikkeld door Salesforce, welke ook de WikiSQL dataset heeft ontwikkeld. Seq2SQL wordt getraind op basis van deze laatst beschreven dataset.

In tegenstelling tot SQLNet, maat Seq2SQL wel gebruik van reïnforcement learning. Het mechanisme krijgt een ssort belong wanneer er een goede query wordt gegenereerd. 

\subsection{Opbouw}

--Opbouw--

\subsection{Uitvoering}

--Uitvoering--

\subsection{Resultaten}

--Resultaten--
