\chapter{Stand van zaken}
\label{ch:stand-van-zaken}

% Tip: Begin elk hoofdstuk met een paragraaf inleiding die beschrijft hoe
% dit hoofdstuk past binnen het geheel van de bachelorproef. Geef in het
% bijzonder aan wat de link is met het vorige en volgende hoofdstuk.

% Pas na deze inleidende paragraaf komt de eerste sectiehoofding.

%Dit hoofdstuk bevat je literatuurstudie. De inhoud gaat verder op de inleiding, maar zal het onderwerp van de bachelorproef *diepgaand* uitspitten. De bedoeling is dat de lezer na lezing van dit hoofdstuk helemaal op de hoogte is van de huidige stand van zaken (state-of-the-art) in het onderzoeksdomein. Iemand die niet vertrouwd is met het onderwerp, weet er nu voldoende om de rest van het verhaal te kunnen volgen, zonder dat die er nog andere informatie moet over opzoeken \autocite{Pollefliet2011}.

%Je verwijst bij elke bewering die je doet, vakterm die je introduceert, enz. naar je bronnen. In \LaTeX{} kan dat met het commando \texttt{$\backslash${textcite\{\}}} of \texttt{$\backslash${autocite\{\}}}. Als argument van het commando geef je de ``sleutel'' van een ``record'' in een bibliografische databank in het Bib\TeX{}-formaat (een tekstbestand). Als je expliciet naar de auteur verwijst in de zin, gebruik je \texttt{$\backslash${}textcite\{\}}.
%Soms wil je de auteur niet expliciet vernoemen, dan gebruik je \texttt{$\backslash${}autocite\{\}}. In de volgende paragraaf een voorbeeld van elk.

%\textcite{Knuth1998} schreef een van de standaardwerken over sorteer- en zoekalgoritmen. Experten zijn het erover eens dat cloud computing een interessante opportuniteit vormen, zowel voor gebruikers als voor dienstverleners op vlak van informatietechnologie~\autocite{Creeger2009}.


Al een geruime tijd zijn bedrijven aan het experimenteren met artificiële intelligentie. Zo worden er tegenwoordig experimenten uitgevoerd met zelfrijdende auto's, robotstofzuigers en spraakherkenning. Spraakherkenning wordt op verschillende manieren gebruikt. Dat kan gaan van het dicteren van een tekst, tot het invoeren van een zoekopdracht in Google, dit via de smartphone. Maar is het ook mogelijk om via spraakherkenning, of in het algemeen tekstherkenning, programmeercode te schrijven door middel van AI? 

Blijkbaar wel! Bedrijven zoals Microsoft en Google zijn al een tijdje bezig met het ontwikkelen van zo'n technologieën. Hier volgt een overzicht van mogelijke technologieën:
\begin{itemize}
	\item Microsoft DeepCoder
	\item Google AutoML
	\item Salesforce - Seq2SQL
	\item SQLNet
	\item Primary Objects - AI Programmer
	\item \ldots
\end{itemize}

\textbf{Microsoft Deepcoder}

Onderzoekers van Microsoft hebben, in samenwerking met de universiteit van Cambridge, een machine learning taal ontwikkeld die, door middel van artificiële intelligentie, zelf code kan schrijven.
Dit mechanisme heeft de naam Deepcoder gekregen.

De bedoeling van Deepcoder is eigenlijk dat een idee, door middel van artificiële intelligentie, omgezet wordt naar bruikbare code en dat binnen een paar seconden. Op dit moment zijn ze al zo ver gevorderd, dat Deepcoder in staat is om basis problemen op te lossen (welke niets meer dan 5 lijnen zijn).

Het is niet zo dat Microsoft Deepcoder de plaats zal innemen van duizenden programmeurs, deze programmeurs zullen de moeilijkste problemen moeten aanpakken, AI zal eerder de veel voorkomende problemen aanpakken. 

Voor dit systeem is geheugen zeer belangrijk. Zo kan het systeem onthouden welke stukken code in het verleden werkten, en andere niet werkten.

AI zal in de toekomst voor verschillende sectoren het eerder vuilere werk opknappen. Programmeurs zullen zich eerder bezighouden met het meer verfijnende werk.

\textbf{Program Synthesis}

Deepcoder werkt door middel van een methode “program synthesis”, wat eigenlijk wil zeggen dat het programma lijnen code steelt van reeds afgewerkte software. Deze manier van werken wordt nu al toegepast door sommige ontwikkelaars(zoals jongeren, script-kiddies, …). 

Er zijn veel werkmogelijkheden met dit soort AI. Het kan sneller en grondiger broncode-databanken doorzoeken om de meest bruikbare code terug te vinden en te gebruiken in de te schrijven programma’s. 

AI is veel slimmer dan de mens, het kan grondiger te werk gaan. DeepCoder gebruikt machine learning om databases van broncode te doorzoeken en de fragmenten te sorteren op basis van hun visie op hun waarschijnlijke bruikbaarheid.

Door dit alles wordt werkende software in een fractie van een seconde geschreven, terwijl oudere mechanismen er minuten aan moeten werken.

Deze technologie kan vele uitwerkingen hebben, zo kan het al foutjes ontdekken in software en deze automatisch oplossen door de foute lijnen te vervangen door werklijnen uit andere programma’s. In de toekomst zou het makkelijker zijn om allerlei routineprogramma’s te maken (info schrappen op websites, facebook foto’s categoriseren, …).

\textbf{Google AutoML}

\textbf{Salesforce - Seq2SQL}

\textbf{SQLNet}

\textbf{AI Programmer}