\chapter{Stand van zaken}
\label{ch:stand-van-zaken}

% Tip: Begin elk hoofdstuk met een paragraaf inleiding die beschrijft hoe
% dit hoofdstuk past binnen het geheel van de bachelorproef. Geef in het
% bijzonder aan wat de link is met het vorige en volgende hoofdstuk.

% Pas na deze inleidende paragraaf komt de eerste sectiehoofding.

%Dit hoofdstuk bevat je literatuurstudie. De inhoud gaat verder op de inleiding, maar zal het onderwerp van de bachelorproef *diepgaand* uitspitten. De bedoeling is dat de lezer na lezing van dit hoofdstuk helemaal op de hoogte is van de huidige stand van zaken (state-of-the-art) in het onderzoeksdomein. Iemand die niet vertrouwd is met het onderwerp, weet er nu voldoende om de rest van het verhaal te kunnen volgen, zonder dat die er nog andere informatie moet over opzoeken \autocite{Pollefliet2011}.

%Je verwijst bij elke bewering die je doet, vakterm die je introduceert, enz. naar je bronnen. In \LaTeX{} kan dat met het commando \texttt{$\backslash${textcite\{\}}} of \texttt{$\backslash${autocite\{\}}}. Als argument van het commando geef je de ``sleutel'' van een ``record'' in een bibliografische databank in het Bib\TeX{}-formaat (een tekstbestand). Als je expliciet naar de auteur verwijst in de zin, gebruik je \texttt{$\backslash${}textcite\{\}}.
%Soms wil je de auteur niet expliciet vernoemen, dan gebruik je \texttt{$\backslash${}autocite\{\}}. In de volgende paragraaf een voorbeeld van elk.

%\textcite{Knuth1998} schreef een van de standaardwerken over sorteer- en zoekalgoritmen. Experten zijn het erover eens dat cloud computing een interessante opportuniteit vormen, zowel voor gebruikers als voor dienstverleners op vlak van informatietechnologie~\autocite{Creeger2009}.


Tegenwoordig is artificiële intelligentie niet meer weg te denken uit onze maatschappij. Denk hierbij maar aan zelfrijdende auto’s of robotstofzuigers die weten waar het vuil is. Naast deze reeds bekendere voorbeelden, kan je met AI nog veel meer doen. Denk hierbij maar aan spraak- en tekstherkenning. Een voorbeeld hiervan is het al sprekend ingeven van zoekopdrachten op Google. Maar bestaat er ook een mogelijkheid om op deze manier, dus via spraak- of tekstherkenning, code te laten genereren?

Verscheidene bedrijven en organisaties zoals Google en Microsoft experimenteren met code generatie door middel van AI. Meestal gaat het hierbij over de generatie van SQL-queries, maar in de toekomst zou het misschien ook mogelijk zijn om bijvoorbeeld Java-code te laten genereren. 

In volgende lijst worden er een aantal voorbeelden weergegeven:
\begin{itemize}
	\item Microsoft DeepCoder
	\item Google AutoML
	\item Salesforce - Seq2SQL
	\item SQLNet
	\item Primary Objects - AI Programmer
	\item Algoritmes voor het transformeren van tekst naar SQL-queries
	\item Microsoft Research - NL2Prog 
	\item \ldots
\end{itemize}
\break
\break
Hier volgt nu informatie over alle opgelijste voorbeelden.

\textbf{Microsoft Deepcoder}

In samenwerking met de universiteit van Cambridge, ontwikkelde Microsoft een mechanisme voor het “zelf schrijven van code”. Hierbij werd gebruik gemaakt van artificiële intelligentie en program synthesis. Het mechanisme kreeg de naam DeepCoder.

Met DeepCoder zouden mensen, ook al hebben ze geen enkele programmeerachtergrond, werkende code kunnen laten schrijven binnen een paar seconden. DeepCoder kan nu al gebruikt worden voor basis problemen op te lossen.

DeepCoder maakt gebruik van Program Synthesis, waarbij DeepCoder eigenlijk lijnen code “steelt” van werkende software. Daarnaast wordt er bruikbare code gezocht op websites zoals StackOverflow om te gebruiken in de te schrijven code. Eigenlijk volgt Program Synthesis de manier van werken van een programmeur (code opzoeken op het internet), maar doet dat veel sneller.

Programmeurs hoeven geen schrik te hebben voor hun job. DeepCoder zou in de toekomst kunnen gebruikt worden om het vuile, herhalende werk op te knappen. Programmeurs zouden de eerder moeilijke taken uitvoeren.

\textbf{Google AutoML}

Naast Microsoft is Google ook bezig met het ontwikkelen van een mechanisme waarbij artificiële intelligentie wordt gebruikt voor de generatie van code. Dit onder de naam “Google AutoML”. Eén van de belangrijkste kenmerken van AutoML, is dat het mechanisme zichzelf altijd tracht te verbeteren. Het is nu al zo ver gekomen dat dit mechanisme betere code kan schrijven dan de programmeurs die het ontwikkeld hebben. 

Google ontwikkelde hun AutoML als oplossing voor het gebrek aan talent bij de AI-programmeurs. De vraag naar gespecialiseerde programmeurs is zo groot geworden, dat AutoML als oplossing werd aangereikt. 

Het systeem voert duizenden simulaties uit om te bepalen welke delen van de code kunnen worden verbeterd, de wijzigingen aan te brengen en de procesadvertentie oneindig door te zetten of totdat het doel is bereikt. 
 
AutoML is nog gloednieuw. Het is ongeveer een jaar geleden voorgesteld. Het vormt de basis voor de volgende generaties in machine-learning.

